\chapter{2007}
\label{cha:2007}

\begin{Problema}{1}
  Calcular el valor de
  \begin{equation}
    \label{eq:1}
    \sqrt{1+3+5+7+\cdots+2003+2005+2007},
  \end{equation}
  donde la suma dentro de la ra�z cuadrada es la suma de todos los
  n�meros impares del 1 al 2007.
\end{Problema}

\begin{Solucion}
  La suma $1+3+\cdots+(2n-1)$ de los primeros $n$ n�meros impares es
  igual a $n^{2}$. Si $2n-1=2007$ entonces $n=1004$, por lo que la
  suma~\ref{eq:1} vale $1004^{2}$.
\end{Solucion}

\begin{Problema}{2}
  Encuentre el volumen de un cono truncado de altura $2$, que tiene
  base inferior de radio $4$ y base superior de radio $3$ (ver la
  figura).
\end{Problema}

\begin{Solucion}
 La f�rmula de volumen del cono es   
\end{Solucion}

\begin{Problema}{3}
  Considere un tri�ngulo de lados $a$, $b$ y $c$. Tome un punto $P$
  cualquiera en el interior del tri�ngulo y desde este punto trace
  segmentos perpendiculares a cada uno de sus lados. Suponga que $x$,
  $y$ y $z$ son las longitudes de estos segmentos perpendiculares a
  los lados $a$, $b$ y $c$, respectivamente. Demuestre que el �rea
  $A$ del tri�ngulo es igual a
  \begin{equation}
    \label{eq:2}
    A=\frac{1}{2}( ax+by+cz). 
  \end{equation}
\end{Problema}

\begin{Solucion}
  
\end{Solucion}

\begin{Problema}{4}
  Del siguiente diagrama calcule de cuantas maneras distintas se puede
  llegar del punto $A$ al punto $B$, respetando las direcciones de las
  flechas.
\[
\SelectTips{xy}{12}
\newdir{|>}{!/4.5pt/@{>}*:(1,-.2)@^{}*:(1,+.2)@_{}}
\entrymodifiers={-<5pt>}
\xymatrix{
 & & & & & &   &       &       &  \bullet   &\bullet\ar@{-|>}[l]_>*+{B}     \\ 
 & & & & & &   &       &       & \bullet\ar@{-|>}[u] &\bullet\ar@{-|>}[l]\ar@{-|>}[u]\\
 & & & & & &   &       &       &  \bullet\ar@{-|>}[u] &\bullet\ar@{-|>}[l]\ar@{-|>}[u]\\
 & & & & & &   &       &       & \bullet \ar@{-|>}[u] &\bullet\ar@{-|>}[l]\ar@{-|>}[u] \\
 & & & & & &   &       &       & \bullet\ar@{-|>}[u] &\bullet\ar@{-|>}[l]\ar@{-|>}[u] \\
 & & & & & &   &       &       & \bullet\ar@{-|>}[u]  &\bullet\ar@{-|>}[l]\ar@{-|>}[u]   \\
 & & & & & &   &       &       & \bullet\ar@{-|>}[u]  &\bullet\ar@{-|>}[l]\ar@{-|>}[u]   \\
 & & & & & &   &      &        & \bullet \ar@{-|>}[u] &\bullet\ar@{-|>}[l]\ar@{-|>}[u]   \\
 & & & & & &   &       &       & \bullet \ar@{-|>}[u] &\bullet\ar@{-|>}[l]\ar@{-|>}[u]   \\
 \bullet\ar@{-|>}[r]\ar@{-}[uuuuuuuuurrrrrrrrr]\ar@{->}[ur]&
 \bullet\ar@{-|>}[r]\ar@{-}[uuuuuuuurrrrrrrr]\ar@{->}[ur]&
 \bullet\ar@{-|>}[r]\ar@{-}[uuuuuuurrrrrrr]\ar@{->}[ur]&
 \bullet\ar@{-|>}[r]\ar@{-}[uuuuuurrrrrr]\ar@{->}[ur]&
 \bullet\ar@{-|>}[r]\ar@{-}[uuuuurrrrr]\ar@{->}[ur]&
 \bullet\ar@{-|>}[r]\ar@{-}[uuuurrrr]\ar@{->}[ur]&
 \bullet\ar@{-|>}[r]\ar@{-}[uuurrr]\ar@{->}[ur]&
 \bullet\ar@{-|>}[r]\ar@{-}[uurr]\ar@{->}[ur]&
 \bullet\ar@{-|>}[r]\ar@{-|>}[ur]&
 \bullet\ar@{-|>}[u]& 
 \bullet \ar@{-|>}[l]\ar@{-|>}[u] \\
\bullet\ar@{-|>}[r]_<*+{A}\ar@{-|>}[u] & \bullet\ar@{-|>}[r]\ar@{-|>}[u]& \bullet\ar@{-|>}[r]\ar@{-|>}[u]& \bullet\ar@{-|>}[r]\ar@{-|>}[u] & \bullet\ar@{-|>}[r]\ar@{-|>}[u]& \bullet \ar@{-|>}[r]\ar@{-|>}[u] & \bullet  \ar@{-|>}[r]\ar@{-|>}[u] &\bullet \ar@{-|>}[r]\ar@{-|>}[u]&\bullet\ar@{-|>}[u]\ar@{-|>}[r] &\bullet\ar@{-|>}[r]\ar@{-|>}[u] & \bullet  \ar@{-|>}[u] \\
 }
\]
\end{Problema}

\begin{Solucion}
  
\end{Solucion}

\begin{Problema}{5}
  Considere la ecuaci�n de segundo grado 
  \begin{equation}
    \label{eq:3}
    x^2-15ax+a^2=0.
  \end{equation}
  Encuentre todos los valores de $a$ de modo que las soluciones $x_1$
  y $x_2$ de esta ecuaci�n satisfacen
  \begin{equation}
    \label{eq:4}
    x_1^2+x_2^2=2007.
  \end{equation}
\end{Problema}

\begin{Solucion}
  
\end{Solucion}

\begin{Problema}{6}
  �De cu�ntas maneras se pueden sacar 10 canicas de una bolsa que
  contiene 7 canicas rojas, 8 azules y 7 verdes, si una vez que se
  sacaron no importa en que orden quedaron?
\end{Problema}

\begin{Solucion}
  
\end{Solucion}

%%% Local Variables: 
%%% mode: latex
%%% TeX-master: "libro"
%%% End: 
